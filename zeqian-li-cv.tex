%-------------------------
% Based on Sourabh Bajaj's template (https://github.com/sb2nov/resume). Credit to him.
%-------------------------

\documentclass[letterpaper,11pt]{article}

\usepackage{latexsym}
\usepackage[empty]{fullpage}
\usepackage{titlesec}
\usepackage{marvosym}
\usepackage[usenames,dvipsnames]{color}
\usepackage{verbatim}
\usepackage{enumitem}
\usepackage[hidelinks]{hyperref}
\usepackage{fancyhdr}
\usepackage[english]{babel}

\pagestyle{fancy}
\fancyhf{} % clear all header and footer fields
\fancyfoot{}
\renewcommand{\headrulewidth}{0pt}
\renewcommand{\footrulewidth}{0pt}

% Adjust margins
\addtolength{\oddsidemargin}{-0.5in}
\addtolength{\evensidemargin}{-0.5in}
\addtolength{\textwidth}{1in}
\addtolength{\topmargin}{-.5in}
\addtolength{\textheight}{1.0in}

\urlstyle{same}

\raggedbottom
\raggedright
\setlength{\tabcolsep}{0in}

% Sections formatting
\titleformat{\section}{
  \vspace{-4pt}\scshape\raggedright\large
}{}{0em}{}[\color{black}\titlerule \vspace{-5pt}]

%-------------------------
% Custom commands
\newcommand{\resumeItem}[2]{
  \item\small{
    \textbf{#1}{ #2 \vspace{-2pt}}
  }
}

\newcommand{\resumeSubheading}[4]{
  \vspace{-1pt}\item
    \begin{tabular*}{0.97\textwidth}[t]{l@{\extracolsep{\fill}}r}
      \textbf{#1} & #2 \\
      \textit{\small#3} & \textit{\small #4} \\
    \end{tabular*}\vspace{-5pt}
}

\newcommand{\resumeSubheadingOneLine}[2]{
  \vspace{-1pt}\item
    \begin{tabular*}{0.97\textwidth}[t]{l@{\extracolsep{\fill}}r}
      \textbf{#1} & #2 \\
    \end{tabular*}\vspace{-5pt}
}

\newcommand{\resumeSubItem}[2]{\resumeItem{#1}{#2}\vspace{-4pt}}

\renewcommand{\labelitemii}{$\circ$}

\newcommand{\resumeSubHeadingListStart}{\begin{itemize}[leftmargin=*]}
\newcommand{\resumeSubHeadingListEnd}{\end{itemize}}
\newcommand{\resumeItemListStart}{\begin{itemize}}
\newcommand{\resumeItemListEnd}{\end{itemize}\vspace{-5pt}}

%-------------------------------------------
%%%%%%  CV STARTS HERE  %%%%%%%%%%%%%%%%%%%%%%%%%%%%


\begin{document}

%----------HEADING-----------------
\begin{tabular*}{\textwidth}{l@{\extracolsep{\fill}}r}
  \textbf{\href{https://zeqianli.github.io/}{\Large Zeqian Li}} &  zeqianli@uchicago.edu \\
\end{tabular*}


%-----------EDUCATION-----------------
\section{Education}
  \resumeSubHeadingListStart
    \resumeSubheading
    {University of Chicago}{United States}
    {Research assistant, Center for Physics of Evolving Systems }{Aug 2020 -- current}
    \resumeSubheading
      {University of Illinois at Urbana-Champaign}{United States}
      {Ph.D in Physics candidate}{Aug 2018 -- current}
    \resumeSubheading
      {Hong Kong Baptist University}{Hong Kong}
      {B.S in Physics (minor in Applied Mathematics)}{Sep 2014 -- July 2018}
  \resumeSubHeadingListEnd



  % Computational capacities of spiking neural networks with critical avalanches, HKBU
  % Research Assistant; Supervisor: Prof. Changsong Zhou                 Jan 2017 – Apr 2018
  % 	Developed a spiking neural network model based on Liquid State Machine and E-I balance neuron model to perform computational tasks. The model showed critical behaviors and its role in neural computation was investigated. 
  % Influence of feedback neuronal connections on signal flow of C. elegans
  % The Chinese Academy of Sciences, Beijing		                      
  % Research Assistant; Supervisors: Dr. Yuhan Chen, Prof. Haijun Zhou
  % 	Studied information flow in C. elegans neural network by analyzing feedback neuronal connections. The study applied a simulated annealing algorithm solving the network minimum feedback arc set (FAS) problem.
  % Evolvement of cell adjacency relationships in C. elegans cell migration, HKBU
  % Research Assistant; Supervisors: Prof. Changsong Zhou, Dr. Zhongying Zhao  Jul 2016 – Mar 2017
  % 	 
  
%-----------EXPERIENCE-----------------
%-----------Publications-----------------
\section{Publications}
\resumeSubHeadingListStart
  \resumeSubheadingOneLine{Closed microbial communities self-organize to persistently cycle carbon}{ \\
  Luis Miguel de Jesús Astacio$^*$, Kaumudi H. Prabhakara$^*$, \textbf{Zeqian Li}, Harry Mickalide, Seppe Kuehn\\
  \textit{Accepted, PNAS, 2021.}} 
\resumeSubHeadingListEnd

%TODO: too dense!!!
\section{Research Experience}
  \resumeSubHeadingListStart
    \resumeSubheading
    {University of Chicago}{United States}
    {Supervisor: Seppe Kuehn}{Aug 2019 - current}
    \resumeItemListStart
      \resumeItem{Yellowstone hot spring microbial mats:} 
      {I am studying the model system Yellowstone hot spring microbial mats using a multi-omics dataset (metagenome, metatranscriptome and single-cell sequencing). Using pangenomics and custom omics tools, I am studying how environments shape microbial community structure and functions. 
      }

      \resumeItem{Evolution of microbial metabolism:}
      {In collaboratioin with Ahmed Selim, we are studying what are essential evolutionary determinants of microbial carbon metabolism using both experiments and computational tools (flux-balance analysis).
      }

      \resumeItem{Pathway splitting in denitrification:}
      {I studied statistical structure of the denitrification pathway in microbial communites and how it was shaped by environments. We hypothesized that environmental pH is an essential determinant. With Kyle Krocker and Karna Gowda, we are performing experiments to test the hypothesis. 
      }

      \resumeItem{Closed microbial communities:}
      {I formulated how pressure change reflects microbial respiration and photosythesis, a crucial part in our closed microbial communities project (de Jesus Astacio et al., \textit{PNAS} 2021). I also helped setting up experimental instruments. 
      }
    %   \resumeItem{Center for Physics of Living Cells (CPLC) lab rotations}
    %   {\hfill \textit{Aug 2018 - Jul 2019 } \\   }
       
    \resumeItemListEnd

    \resumeSubheading
      {Hong Kong Baptist University}{Hong Kong}
      {Supervisor: Changsong Zhou}{Jul 2016 - Jul 2018}
      \vspace{1mm}
  

      I studied various biological systems (C. elegans development, C. elegans neural systems, biological neural networks) using theoretical and computational tools stem from statistical physics.
    %   \resumeItemListStart
    %     \resumeItem{Computational capacities of spiking neural networks with critical avalanches}{}
    %       % {\\
    %       % We developed a spiking neural network model to perform computational tasks under supervision. The model, inspired by Liquid State Machine and excitation-inhibition balanced neurons, showed critical behaviors. We studied roles of criticality in neural computation.}
    %     \resumeItem{Cell adjacency relationships in C. elegans cell migration}{}
    %       %{\\
    %       %We studied C. elegans’ early embryonic development by investigating cell adjacency relationships. We showed that cell contacts were deterministic across wild-type individuals.}
    %     \resumeItem{Feedback connections on C. elegans neural signal flow}{}
    %   \resumeItemListEnd
        
  \resumeSubHeadingListEnd



% %-----------TEACHING-----------------
% \section{Teaching Experience}
% \resumeSubHeadingListStart

%   \resumeSubheadingOneLine{Hong Kong Baptist University}{Teaching assistant for introductory physics courses.}
  
% \resumeSubHeadingListEnd

\section{Honors and Rewards}
\resumeSubHeadingListStart
\vspace{-1pt}\item
    \begin{tabular*}{0.97\textwidth}[t]{l@{\extracolsep{\fill}}r}
      \textbf{Center for Physics of Living Cells (CPLC) Fellow} & UIUC, 2018-2020 \\
    \end{tabular*}\vspace{-5pt}
\vspace{-1pt}\item
    \begin{tabular*}{0.97\textwidth}[t]{l@{\extracolsep{\fill}}r}
      \textbf{HKSAR Government Scholarship} & Hong Kong, 2015-2018 \\
    \end{tabular*}\vspace{-5pt}
\vspace{-1pt}\item
\begin{tabular*}{0.97\textwidth}[t]{l@{\extracolsep{\fill}}r}
  \textbf{Scholastic Award} & Hong Kong Baptist University, 2018 \\
\end{tabular*}\vspace{-5pt}

\resumeSubHeadingListEnd

%----------OTHER ACTIVITIES-----------
\section{Others}
\resumeSubHeadingListStart
  \resumeSubheadingOneLine{ICTP Spring College on the Physics of Complex Systems (2018)}{ICTP, Triest, Italy \\ Took five graduate courses with grade E (excellent).}
  \resumeSubheadingOneLine{Teaching experience}{\\ Teaching assistant for introductory physics courses in college and graduate school.}
  % \resumeItemListStart
  % \resumeItem{Spring College on the Physics of Complex Systems}
  % {\hfill \textit{Feb 2018 - Mar 2018}\\Took five graduate courses (grade: E (excellent)): Nonequilibrium behavior of quantum statistical systems (Maurizio Fagotti), Statistics of extremes in correlated systems (Gregory Schehr), Hierarchical inference (C. Mathys), Reinforcement learning (Antonio Celani), Polymer physics of chromosome folding (Angelo Rosa, Mario Nicodemi)}
  % \resumeItemListEnd
  
\resumeSubHeadingListEnd

%--------PROGRAMMING SKILLS------------
\section{Skills}
 \resumeSubHeadingListStart
  \resumeSubItem{Experiments:}{common microbiology lab operation and assays.}
  \resumeSubItem{Bioinformatics:}{common omics tools, 16s data analysis, metagenome data analysis.}
  \resumeSubItem{Programming:}{Python, Java, Javascript, C/C++, Matlab, Bash, \LaTeX.}
 \resumeSubHeadingListEnd


%-------------------------------------------
\end{document}
